\documentclass{article}
\title{Chapter 08}
\author{Newton Ni}

\usepackage{bussproofs}
\usepackage{amsmath}
\usepackage{amssymb}
\usepackage{amsthm}

% Set cardinality: |#1|
\newcommand{\cardinality}[1]{\lvert#1\rvert}

% Set: { #1 }
\newcommand{\set}[1]{\{\ #1\ \}}

% Set comprehension: { #1 | #2 }
\newcommand{\comp}[2]{\set{#1\ \mid\ #2}}

% t₁, t₂, t₃, ...
\newcommand{\term}[1]{\texttt{t\textsubscript{#1}}}

% v₁, v₂, v₃, ...
\newcommand{\val}[1]{\texttt{v\textsubscript{#1}}}

\renewcommand{\ss}[2]{#1 \longrightarrow #2}
\renewcommand{\bs}[2]{#1 \Downarrow #2}

% Monospace
\newcommand{\ms}[1]{\texttt{#1}}

\newcommand{\LabelR}[1]{\RightLabel{\textsc{#1}}}

\theoremstyle{remark}
\newtheorem*{case}{Case}

\begin{document}
\maketitle

\section{8.2.3}

    \textit{Prove that every subterm of a well-typed term is well typed.}

    \begin{proof}
        By induction on the typing derivation of term \term{}. Our inductive hypothesis is:
        $$H(\term{}): \term{} \textit{ is well-typed} \implies \textit{all subterms of } \term{} \textit{ are well-typed}$$

        We proceed by case analysis on the final step of the derivation.
        \begin{case}[\ms{true}, \ms{false}, 0]
            These terms are well-typed by the inversion lemma, and have no subterms.
        \end{case}
        \begin{case}[\ms{if \term{1} then \term{2} else \term{3}}]
            By the inversion lemma, this term is well-typed, and we have three well-typed subterms:
            \begin{itemize}
                \item $\term{1} : \ms{Bool}$
                \item $\term{2} : \ms{R}$
                \item $\term{3} : \ms{R}$
            \end{itemize}
            We can apply the inductive hypothesis to each.
        \end{case}
        \begin{case}[\ms{succ \term{1}}, \ms{pred \term{1}}, \ms{iszero \term{1}}]
            These cases are analagous to the previous case, except they
            only have one subterm each.
        \end{case}
    \end{proof}

\section{8.3.4}

    \textit{Restructure} [the \textsc{Preservation}] \textit{proof so that it goes by}
    \textit{induction on evaluation derivations rather than typing derivations.}

    \begin{proof}
        By induction on the derivation of $\ss{\term{}}{\term{}'}$. Our inductive hypothesis is:
        $$H(\term{}): \term{} : T\ \land\ \ss{\term{}}{\term{}'} \implies \term{}' : T$$
        We proceed by case analysis on the last step of the derivation. In all cases,
        we have that $\term{} : T$.

        \begin{case}[\textsc{E-IfTrue}]
            Here, $\ss{\term{}}{\term{}'}$ is:
            $$\ss{\ms{if true then \term{2} else \term{3}}}{\term{2}}$$
            By the inversion lemma, $\term{} : T \implies \term{2} : T$ as desired.
        \end{case}
        \begin{case}[\textsc{E-IfFalse}]
            This case is analagous to the previous one.
        \end{case}

        \begin{case}[\textsc{E-If}]
            Here, we have:
            $$\ss{\ms{if \term{1} then \term{2} else \term{3}}}{\ms{if \term{1}' then \term{2} else \term{3}}}$$
            $$\ss{\term{1}}{\term{1}'}$$
            By the inversion lemma, we have:
            \begin{itemize}
                \item $\term{1} : \ms{Bool}$
                \item $\term{2} : \ms{T}$
                \item $\term{3} : \ms{T}$
            \end{itemize}
            By the inductive hypothesis, we also have $\term{1}' : \ms{Bool}$.
            We conclude $\term{}' : \ms{T}$ by \textsc{T-If}.
        \end{case}

        \begin{case}[\textsc{E-Succ}]
            Here, we have:
            $$\ss{\ms{succ \term{1}}}{\ms{succ \term{1}'}}$$
            $$\ss{\term{1}}{\term{1}'}$$
            By the inversion lemma, we have $\term{1} : \ms{Nat}$.
            By the inductive hypothesis, we have $\term{1}' : \ms{Nat}$.
            We conclude that $\ms{succ \term{1}'} : \ms{Nat}$ by \textsc{T-Succ}.
        \end{case}
        \begin{case}[\textsc{E-Pred}, \textsc{E-IsZero}]
            These cases are analagous to the previous case.
        \end{case}

        \begin{case}[\textsc{E-PredZero}, \textsc{E-IsZeroZero}]
            These cases follow directly from the inversion lemma and typing
            rules \textsc{T-Zero} and \textsc{T-True}, respectively.
        \end{case}

        \begin{case}[\textsc{E-PredSucc}]
            Here, we have:
            $$\ss{\ms{pred (succ \term{1})}}{\term{1}}$$
            By two applications of the inversion lemma, we have:
            \begin{itemize}
                \item $\ms{succ \term{1}} : \ms{Nat}$
                \item $\ms{\term{1}} : \ms{Nat}$
            \end{itemize}
        \end{case}
        \begin{case}[\textsc{E-IsZeroSucc}]
            This case is analagous to the previous case.
        \end{case}

    \end{proof}

\section{8.3.5}

    \textit{The evaluation rule \textsc{E-PredZero} (Figure 3--2) is a bit counterintuitive:}
    \textit{we might feel that it makes more sense for the predecessor of zero to be undefined,}
    \textit{rather than being defined to be zero. Can we achieve this simply by removing the}
    \textit{rule from the definition of single-step evaluation?} \\

    No, because the term $\ms{pred 0}$ would be well-typed by \textsc{T-Zero} and \textsc{T-Pred},
    but would be stuck, violating the progress property.

\section{8.3.6}

    \textit{Having seen the subject reduction property, it is reasonable to wonder whether}
    \textit{the opoosite property---\text{subject expansion}---also holds. Is it always the}
    \textit{case that, if} $\ss{\term{}}{\term{}'}$ \textit{and} $\term{}' : \ms{T}$ \textit{, then}
    $\term{} : \ms{T}$? \textit{If so, prove it. If not, give a counterexample.} \\

    No, since ill-typed terms can also take a step. For example:
    $$\ss{\ms{if true then 0 else true}}{0}$$

\section{8.3.7}

    \textit{Suppose our evaluation relation is defined in the big-step style, as in exercise}
    \textit{3.5.17. How should the intuitive property of type safety be formalized?} \\

    Type preservation remains the same, but progress is modified to state that all well-typed terms
    evaluate to a value.

\section{8.3.8}

    \textit{Suppose our evaluation relation is augmented with rules for reducing nonsensical}
    \textit{terms to an explicit} \ms{wrong} \textit{state, as in Exercise 3.5.16. Now how}
    \textit{should type safety be formalized?} \\

    Type preservation remains the same, but progress is no longer useful, as terms can always
    take a step (sometimes to \ms{wrong}).

\end{document}
