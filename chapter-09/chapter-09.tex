\documentclass{article}
\title{Chapter 09}
\author{Newton Ni}

\usepackage{bussproofs}
\usepackage{amsmath}
\usepackage{amssymb}
\usepackage{amsthm}
\usepackage{turnstile}

% Set cardinality: |#1|
\newcommand{\cardinality}[1]{\lvert#1\rvert}

% Set: { #1 }
\newcommand{\set}[1]{\{\ #1\ \}}

% Set comprehension: { #1 | #2 }
\newcommand{\comp}[2]{\set{#1\ \mid\ #2}}

% t₁, t₂, t₃, ...
\newcommand{\term}[1]{\texttt{t\textsubscript{#1}}}

% v₁, v₂, v₃, ...
\newcommand{\val}[1]{\texttt{v\textsubscript{#1}}}

% T₁, T₂, T₃, ...
\newcommand{\ty}[1]{\texttt{T\textsubscript{#1}}}

% x₁, x₂, x₃, ...
\newcommand{\var}[1]{\texttt{x\textsubscript{#1}}}

% T -> T
\newcommand{\fn}[2]{#1$\rightarrow$#2}

\renewcommand{\ss}[2]{#1 \longrightarrow #2}
\renewcommand{\bs}[2]{#1 \Downarrow #2}

% Monospace
\newcommand{\ms}[1]{\texttt{#1}}

\newcommand{\LabelR}[1]{\RightLabel{\textsc{#1}}}

\theoremstyle{remark}
\newtheorem*{case}{Case}

\begin{document}
\maketitle

\section{9.2.1}

    \textit{The pure simply typed lambda-calculus with no base types is actually} degenerate\textit{,}
    \textit{in the sense that it has no well-typed terms at all. Why?} \\

    There are no base cases in the definition of types. There are only function types
    $\ty{} ::= \ty{} \rightarrow \ty{}$, which cannot generate a finite type.

\section{9.2.2}

    \textit{Show (by drawing derivation trees) that the following terms have the indicated types:}

    $$\texttt{f:Bool} \rightarrow \texttt{Bool} \vdash \texttt{f (if false then true else false) : Bool}$$

    \begin{prooftree}
        \LabelR{\texttt{Var}}
        \AxiomC{\texttt{f:\fn{Bool}{Bool}} $\in$ \texttt{f:\fn{Bool}{Bool}}}
        \UnaryInfC{\texttt{f:\fn{Bool}{Bool}} $\vdash$ \texttt{f:\fn{Bool}{Bool}}}

        \LabelR{\texttt{F}}
        \AxiomC{}
        \UnaryInfC{$\vdash$ \texttt{false:Bool}}

        \LabelR{\texttt{T}}
        \AxiomC{}
        \UnaryInfC{$\vdash$ \texttt{true:Bool}}

        \LabelR{\texttt{F}}
        \AxiomC{}
        \UnaryInfC{$\vdash$ \texttt{false:Bool}}

        \LabelR{\texttt{If}}
        \TrinaryInfC{\texttt{if false then true else false:Bool}}

        \LabelR{\texttt{App}}
        \BinaryInfC{\texttt{f (if false then true else false):Bool}}
    \end{prooftree}

    $$
        \texttt{f:Bool}\rightarrow\texttt{Bool}
        \vdash
        \lambda\var{}\texttt{:Bool. f (if x then false else x):Bool}\rightarrow\texttt{Bool}
    $$

    \begin{prooftree}
        \LabelR{Var}
        \AxiomC{$\ldots$}
        \UnaryInfC{$\ldots\vdash\texttt{f:Bool}\rightarrow\texttt{Bool}$}

        \LabelR{Var}
        \AxiomC{$\ldots$}
        \UnaryInfC{$\ldots\vdash\texttt{x:Bool}$}

        \LabelR{\texttt{F}}
        \AxiomC{}
        \UnaryInfC{$\vdash$ \texttt{false:Bool}}

        \LabelR{Var}
        \AxiomC{$\ldots$}
        \UnaryInfC{$\ldots\vdash\texttt{x:Bool}$}

        \LabelR{If}
        \TrinaryInfC{$\ldots\vdash\texttt{if x then false else x:Bool}$}

        \LabelR{App}
        \BinaryInfC{$\texttt{f:Bool}\rightarrow\texttt{Bool},\texttt{x:Bool}\vdash\texttt{f (if x then false else x):Bool}$}

        \LabelR{Abs}
        \UnaryInfC{$\texttt{f:Bool}\rightarrow\texttt{Bool}\vdash\ldots\texttt{:Bool}\rightarrow\texttt{Bool}$}
    \end{prooftree}

\section{9.2.3}

    \textit{Find a context} $\Gamma$ \textit{under which the term} \texttt{f x y} \textit{has type} \texttt{Bool}\textit{.}
    \textit{Can you give a simple description of the set of} all \textit{such contexts?} \\

    The type of $f$ could be any \fn{\ty{1}}{\fn{\ty{2}}{\texttt{Bool}}}, where $x:\ty{1}$ and $y:\ty{2}$
    and the latter two types are unconstrained.

\section{9.3.2}

    \textit{Is there any context} $\Gamma$ \textit{and type} \ty{} \textit{such that} $\Gamma\vdash\texttt{x x:\ty{}}$?
    \textit{If so, give} $\Gamma$ \textit{and} \ty{} \textit{and show a typing derivation; if not, prove it.} \\

    No. Suppose \var{}: \fn{\ty{1}}{\ty{2}}. To successfully apply it to itself, \textsc{T-App} requires
    that \var{}: \ty{1}. But there is no finite type such that $\ty{1} = \ty{1} \rightarrow \ty{2}$.

\section{9.3.9}

    \textit{Prove the preservation theorem:}
    $
        \Gamma\vdash\term{}:\ty{} \land \ss{\term{}}{\term{}'}
        \implies
        \Gamma\vdash\term{}':\ty{}
    $.

    \begin{proof}
        By induction on the small-step derivation for $\ss{\term{}}{\term{}'}$. Our induction hypothesis is:
        $$H(\ss{\term{}}{\term{}'}): \Gamma\vdash\term{}:\ty{} \implies \Gamma\vdash\term{}':\ty{}$$

        \begin{case}[\textsc{E-App1}]
            Here we have $\term{} = \term{1}\ \term{2}$, where:
            \begin{enumerate}
                \item $\ss{\term{1}\ \term{2}}{\term{1}'\ \term{2}}$
                \item $\ss{\term{1}}{\term{1}'}$
            \end{enumerate}
            By the inversion lemma, we have:
            \begin{enumerate}
                \item[3.] $\Gamma \vdash$ \term{1}: \fn{\ty{1}}{\ty{}}
                \item[4.] $\Gamma \vdash$ \term{2}: \ty{1}
            \end{enumerate}
            We apply the induction hypothesis to (2) and (3) to obtain
            \begin{enumerate}
                \item[5.] $\Gamma \vdash \term{1}'$: \fn{\ty{1}}{\ty{}}
            \end{enumerate}
            We then apply \textsc{T-App} to (4) and (5) to conclude that $\term{}' = \term{1}'\ \term{2}: \ty{}$, as desired.
        \end{case}

        \begin{case}[\textsc{E-App2}]
            This case is analagous to above, except we apply the induction hypothesis to $\term{2}$ instead.
        \end{case}

        \begin{case}[\textsc{E-AppAbs}]
            Here we have:
            \begin{enumerate}
                \item $\term{} = (\lambda \var{}.\ \term{12}) \val{2}$
                \item $\term{}' = [\var{} \mapsto \val{2}]\ \term{12}$
            \end{enumerate}
            By the inversion lemma, we have:
            \begin{enumerate}
                \item[3.] $\Gamma \vdash \lambda \var{}.\ \term{12}$: \fn{\ty{1}}{\ty{}}
                \item[4.] $\Gamma \vdash \val{2}: \ty{1}$
            \end{enumerate}
            By \textsc{T-Abs}, we have:
            \begin{enumerate}
                \item[5.] $\Gamma,\ \var{}:\ty{1} \vdash \term{12}$: \ty{}
            \end{enumerate}
            By applying the substitution lemma to (5) and (4), we have:
            \begin{enumerate}
                \item[6.] $\Gamma \vdash [\var{} \mapsto \val{2}]\ \term{12} : \ty{}$ 
            \end{enumerate}
            As desired.
        \end{case}
    \end{proof}

\section{9.3.10}

    \textit{In Exercise 8.3.6 we investigated the} subject expansion \textit{property for our simple calculus}
    \textit{of typed arithmetic expressions. Does it hold for the `functional part' of the simply typed lambda-calculus?}
    \textit{That is, suppose} \term{} \textit{does not contain any conditional expressions. Do} $\ss{\term{}}{\term{}'}$
    \textit{and} $\Gamma \vdash \term{}' : \ty{}$ \textit{imply} $\Gamma \vdash \term{} : \ty{}$? \\

    TODO

\section{9.4.1}

    \textit{Which of the rules for the type} \texttt{Bool} \textit{in Figure 8-1 are introduction rules and}
    \textit{which are elimination rules? What about the rules for} \texttt{Nat} \textit{in figure 8-2?} \\

    For \texttt{Bool}, \textsc{T-True} and \textsc{T-False} are introduction rules,
    while \textsc{T-If} is an elimination rule.

    For \texttt{Nat}, \textsc{T-Zero}, \textsc{T-Succ}, and \textsc{T-Pred} are introduction rules,
    while \textsc{T-IsZero} is an elimination rule.

\end{document}
