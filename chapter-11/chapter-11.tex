\documentclass{article}
\title{Chapter 11}
\author{Newton Ni}

\usepackage{bussproofs}
\usepackage{amsmath}
\usepackage{amssymb}
\usepackage{amsthm}
\usepackage{turnstile}

% Set cardinality: |#1|
\newcommand{\cardinality}[1]{\lvert#1\rvert}

% Set: { #1 }
\newcommand{\set}[1]{\{\ #1\ \}}

% Set comprehension: { #1 | #2 }
\newcommand{\comp}[2]{\set{#1\ \mid\ #2}}

% t₁, t₂, t₃, ...
\newcommand{\term}[1]{\texttt{t\textsubscript{#1}}}

% v₁, v₂, v₃, ...
\newcommand{\val}[1]{\texttt{v\textsubscript{#1}}}

% l₁, l₂, l₃, ...
\newcommand{\lbl}[1]{\texttt{l\textsubscript{#1}}}

% p₁, p₂, p₃, ...
\newcommand{\pat}[1]{\texttt{p\textsubscript{#1}}}

% T₁, T₂, T₃, ...
\newcommand{\ty}[1]{\texttt{T\textsubscript{#1}}}

% x₁, x₂, x₃, ...
\newcommand{\var}[1]{\texttt{x\textsubscript{#1}}}

% T -> T
\newcommand{\fn}[2]{#1$\rightarrow$#2}

\renewcommand{\ss}[2]{#1 \longrightarrow #2}
\renewcommand{\bs}[2]{#1 \Downarrow #2}

% Monospace
\newcommand{\ms}[1]{\texttt{#1}}

\newcommand{\LabelR}[1]{\RightLabel{\textsc{#1}}}

\theoremstyle{remark}
\newtheorem*{case}{Case}

\begin{document}
\maketitle

\section{11.4.1}

    \textit{Show how to formulate ascription as a derived form. Prove that the ``official'' typing}
    \textit{and evaluation rules given here correspond to your definition in a suitable sense.} \\

    We can formulate ascription using the following transformation \ms{Lower}:
    \begin{align*}
        \ms{Lower(t as T)} &= (\lambda \var{} : \ty{}.\ \var{})\ \term{} \\
        \ms{Lower(t)} &= \term{}
    \end{align*}

    We wish to prove the following:
    \begin{align}
        \Gamma \vdash \term{} : \ty{} &\iff \Gamma \vdash \ms{Lower(t)} : \ty{} \\
        \ss{\term{}}{\term{}'} &\iff \ss{\ms{Lower(t)}}{\ms{Lower(\term{}')}}
    \end{align}

    We induct on terms \term{}. For both (1) and (2), all cases except \ms{t = \term{1} as T}
    follow from straightforward induction. For the forward case of (1), we know the following:

    \begin{enumerate}
        \item[1.F.1] $\Gamma \vdash \term{1} :\ \ty{}$
        \item[1.F.2] $\ms{Lower(\term{1} as T)} = (\lambda \var{} : \ty{}.\ \var{})\ \term{1}$
    \end{enumerate}

    The conclusion follows from applying \textsc{T-App} to (1.F.2), after type-checking
    the identity function with \textsc{T-Abs}, \textsc{T-Var}, and (1.F.1) (proof tree omitted). \\

    For the backward case of (1), we know:
    \begin{enumerate}
        \item[1.B.1] $\Gamma \vdash \ms{Lower(\term{1} as T)} : \ty{}$
        \item[1.B.2] $\Gamma \vdash (\lambda \var{} : \ty{}.\ \var{})\ \term{1} : \ty{}$
    \end{enumerate}

    Applying \textsc{T-Abs}, \textsc{T-Var} in reverse (probably need lemma to prove
    this is okay, as there is only one typing judgment each for these syntactic forms)
    yields (1.B.3): $\term{1} : \ty{}$, which is enough to finish with \textsc{T-Ascribe}. \\

    For the forward case of (2), there are two sub-cases: when \term{1} is a value or not.
    When it is a value \val{1}, we apply \textsc{E-Ascribe} to the left-hand side and have
    $\ss{\ms{\val{1} as \ty{}} }{\val{1}}$. On the right-hand side, we want to show that
    $\ss{(\lambda \var{} : \ty{}.\ \var{})\ \val{1}}{\val{1}}$. This follows directly from
    \textsc{E-AppAbs}.

    When \term{1} is not a value, we apply \textsc{E-Ascribe1} to the left-hand side and
    get $\ss{\ms{\term{1} as \ty{}}}{\term{1}'}$. On the right-hand side, since the identity
    function is a value, the conclusion follows from applying \textsc{E-App2} and the determinancy
    of small-step evaluation. \\

    For the backward case of (2), we again distinguish when \term{1} is a value or not,
    and the logic is similar to above. \\

    \textit{Suppose that, instead of the pair of evaluation rules \textsc{E-Ascribe} and \textsc{E-Ascribe1},}
    \textit{we had given an ``eager'' rule that throws away an ascription as soon as it is reached. Can}
    \textit{ascription still be considered as a derived form?} \\

    It depends on the evaluation strategy for the base lambda calculus. For the evaluation rules
    in the textbook, it would no longer be a derived form, as \textsc{E-AscribeEager} would diverge
    from the behavior of \textsc{E-App2} when \term{1} is not a value. But if we changed the evaluation
    strategy to apply functions to their argument eagerly as well, then this would be a derived form.

\section{11.5.2}

    Not sure how to define ``good idea'', but this seems to tangle the evaluation and typing judgments
    (requiring substitions in order to type-check), and require more computation. I suppose errors would
    also be harder to track to their source spans, if the error arises within substituted code?

\section{11.8.2}

    \textit{We can add a simple form of pattern matching to an untyped lambda calculus with records}  
    \textit{by adding a new syntactic category of} patterns . . . \textit{Give typing rules for the}
    \textit{new constructs (making any changes to the syntax you feel are necessary in the process).} \\

    \begin{prooftree}
        \LabelR{T-Match}
        \AxiomC{$match(p, \val{1}) = \sigma$}
        \AxiomC{$\Gamma \vdash \sigma\ \term{2} : \ty{}$}
        \BinaryInfC{$\Gamma \vdash match(\pat{}, \val{1})\ \term{2}: \ty{}$}
    \end{prooftree}

    \textit{Sketch a proof of type preservation and progress for the whole calculus.} \\

\section{11.11.1}

    We can define factorial as:
    \begin{align*}
        &\ms{fix} \\
        &\lambda \ms{factorial}: \ms{int} \to \ms{int}. \\
        &\lambda \var{}: \ms{int}. \\
        &\ms{if}\ \var{}\ =\ 0\ \ms{then}\ 1\ \ms{else}\ \ms{factorial}\  (\var{} - 1) * \var{}
    \end{align*}

\section{11.11.2}

    We can rewrite factorial as:
    \begin{align*}
        &\ms{letrec factorial: int $\to$ int} = \\
        &\lambda \var{}: \ms{int}. \\
        &\ms{if}\ \var{}\ =\ 0\ \ms{then}\ 1\ \ms{else}\ \ms{factorial}\  (\var{} - 1) * \var{}
    \end{align*}

\section{11.12.1}

    \textit{Verify that the progress and preservation theorems hold for the simply typed lambda-calculus}
    \textit{with booleans and lists.}

\section{11.12.2}

    \textit{The presentation of lists here includes many type annotations that are not really needed,}
    \textit{in the sense that the typing rules can easily derive the annotations from context. Can} all
    \textit{the type annotations be deleted?} \\
    
    The type annotation for \ms{nil} cannot be deleted, because there's no argument to derive the type from.
    But all of the other syntactic forms for lists (\ms{cons}, \ms{isnil}, \ms{head}, \ms{tail}) can check
    the type of their argument, so their type annotations are unnecessary.

\end{document}
