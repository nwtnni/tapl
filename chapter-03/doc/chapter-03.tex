\documentclass{article}
\title{Chapter 03}
\author{Newton Ni}

\usepackage{bussproofs}
\usepackage{amsmath}
\usepackage{amssymb}
\usepackage{amsthm}

% Set cardinality: |#1|
\newcommand{\cardinality}[1]{\lvert#1\rvert}

% Set: { #1 }
\newcommand{\set}[1]{\{\ #1\ \}}

% Set comprehension: { #1 | #2 }
\newcommand{\comp}[2]{\set{#1\ \mid\ #2}}

% t₁, t₂, t₃, ...
\newcommand{\term}[1]{\texttt{t\textsubscript{#1}}}

% Monospace
\newcommand{\ms}[1]{\texttt{#1}}

\theoremstyle{remark}
\newtheorem*{case}{Case}

\begin{document}
\maketitle

\section{3.2.4}

    \textit{How many elements does $S_3$ have?}

    \begin{align*}
        \cardinality{S_0} &= 0 \\
        \cardinality{S_1} &= 3 \\
        \cardinality{S_2} &= 3^3 + 3 \cdot 3 + 3 = 39 \\
        \cardinality{S_3} &= 39^3 + 39 \cdot 3 + 3 = 59439
    \end{align*}

\section{3.2.5}

    \textit{Show that the sets $S_i$ are }cumulative\textit{---that is,}
    \textit{for each i we have $S_i \subseteq S_{i + 1}$.}

    \begin{proof}
        By induction on $i$. Our inductive hypothesis is:
        \begin{align*}
            H(i): S_i \subseteq S_{i + 1}
        \end{align*}

        \begin{case}[$i = 0$]
            By definition, $S_0 = \emptyset$, which is a subset of any set.
        \end{case}

        \begin{case}[$i > 0$]
            We want to show:
            \begin{align*}
                \forall t.\ t \in S_i \implies t \in S_{i + 1}
            \end{align*}

            By definition of $S_i$, $t$ must belong to one of three sets:

            \begin{itemize}
                \item{\set{\ms{true},\ \ms{false},\ \ms{zero}}}

                These are in $S_{i + 1}$ by definition.

                \item{\comp{\ms{succ \term{1}},\ \ms{pred \term{1}},\ \ms{iszero \term{1}}}{\term{1}\ $\in S_{i - 1}$}}

                By the inductive hypothesis, we know that $\term{1} \in S_i$.\\
                Then $\ms{succ \term{1}}, \ms{pred \term{1}}, \ms{iszero \term{1}} \in S_{i + 1}$, by construction.

                \item{\comp{\ms{if \term{1} then \term{2} else \term{3}}}{\term{1},\ \term{2},\ \term{3}\ $\in S_{i - 1}$}}

                Similarly, by the inductive hypothesis, $\term{1}, \term{2}, \term{3} \in S_i$,\\
                and $\ms{if \term{1} then \term{2} else \term{3}} \in S_{i + 1}$ by construction.
            \end{itemize}
        \end{case}
    \end{proof}
\end{document}
